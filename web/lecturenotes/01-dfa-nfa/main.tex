\documentclass[xcolor=table]{beamer}

\def\print{0}


\def\COURSENAME{Computability and Complexity}
\def\COURSECODE{CSCI 387}
\def\YEAR{F22}

\definecolor{lightgreen}{cmyk}{.25,0,.25,0}
\definecolor{black}{rgb}{0,0,0}
\definecolor{dgray}{rgb}{0.1,0.1,0.1}
\definecolor{lgray}{rgb}{0.5,0.5,0.5}
\definecolor{dblue}{rgb}{.2,.4,.6}
\definecolor{lblue}{rgb}{.2,0,.7}
\definecolor{llblue}{rgb}{.2,0,.6}
\definecolor{gold}{rgb}{0.8512,0.6042,0.0000}
\definecolor{lred}{rgb}{1.0000,0.2013,0.1405}
\definecolor{orange}{rgb}{1,0.3,0}
\definecolor{dgreen}{rgb}{.01,.6,.06}
\definecolor{lgreen}{rgb}{0.2610,0.8730,0.2875}
\definecolor{peachpuff}{rgb}{1.0000,0.7197,0.5181}
\definecolor{cornflowerblue}{rgb}{0.0859,0.2912,0.8708}
\definecolor{deepskyblue}{rgb}{0.0000,0.5227,1.0000}
\definecolor{lightskyblue}{rgb}{0.1946,0.6324,1.0000}
\definecolor{darkturquoise}{rgb}{0.0000,0.6324,0.6861}
\definecolor{darkseagreen}{rgb}{0.2264,0.5227,0.2498}
\definecolor{lemon}{rgb}{0.8512,0.8416,0.5181}
\definecolor{dred}{rgb}{0.9,0,0}
\definecolor{dbrown}{rgb}{.6,0,.1}

\setbeamercolor{alerted text}{fg=deepskyblue}
\setbeamersize{text margin left=5mm,text margin right=5mm}

\mode<presentation>
{
\ifnum\print=1
  \setbeamertemplate{background canvas}

  \usetheme{default}
  \usecolortheme{rose}
\else
  \setbeamertemplate{background canvas}[vertical shading][bottom=white!10,top=blue!15]
%  \usetheme{Warsaw}
  \usecolortheme[named=dbrown]{structure}
  \usetheme{Warsaw}
\fi
  \usefonttheme[onlysmall]{structurebold}
}


\mode<article> % only for the article version
{
  \usepackage{fullpage}
  \usepackage{hyperref}
}


\usepackage[all,arc,dvips]{xypic}
\usepackage{pgf,pgfarrows,pgfnodes,pgfautomata,pgfheaps,pgfshade}
\usepackage{amsmath,amssymb}
\usepackage[latin1]{inputenc}
\usepackage{diagbox}
%\usepackage{booktabs}
%\usepackage{colortbl}
%\usepackage[english]{babel}
%\usepackage[thai]{babel}
%\usepackage{times}
\setbeamercovered{dynamic}
%\usepackage{pgfpages}
%\pgfpagelayout{2 on 1}{a4paper,border shrink=5mm}
%\mode<handout>{\setbeamercolor{background canvas}{bg=black!5}}

%\newcolumntype{h}{>{\columncolor{gray!50}}l}

\newcommand\gcc[1]{\cellcolor{gray!20}{#1}}

\newcommand{\Z}{\mathbf{Z}}
\newcommand{\R}{\mathbf{R}}
\newcommand{\N}{\mathbf{N}}
\newcommand{\T}{\mathbf{T}}
\newcommand{\F}{\mathbf{F}}
\newcommand{\Q}{\mathbf{Q}}
\newcommand{\bits}{\{0,1\}}
\newcommand{\Colon}{\;:\;}

\newcommand{\bheading}[1]{\vspace{10pt} \noindent \textbf{#1}}

\newenvironment{problemstmt}[1]{\begin{rm}
  \vspace{.3in}
  \noindent \textbf{Problem #1.}}
  {\end{rm}
  \vspace{.3in}
  \hrule}




\title{Regular Languages}
\author{Chanathip Namprempre}
\institute{
  Computer Science \\
  Reed College}
\date{}
\subject{\COURSENAME}

\begin{document}



%%%%%%%%%%%%%%%%%%%%%%%%%%%%%%%%%%%%%%%%%%%%%%%%%%%%%%%%%%%%
%%             CONTENTS START HERE


\frame{\titlepage}

%% \section<presentation>*{Outline}

%% \begin{frame}
%%   \frametitle{Outline}
%%   \tableofcontents[part=1]
%% \end{frame}

\AtBeginSection[]
{
  \begin{frame}<beamer>
    \frametitle{Outline}
    \tableofcontents[current,currentsection]
  \end{frame}
}


\setbeamertemplate{footline}[frame number]
\setbeamertemplate{navigation symbols}{}

\part<presentation>{Main Talk}

\section[Theme]{What this course is about}

\begin{frame}
  \frametitle{Computational Theory}

  \begin{itemize}
    \item limitations of computers
    \item complexity
  \end{itemize}

\end{frame}

\begin{frame}

  In this class, it is helpful to view \alert{\textbf{problems}} as
  specification of input-output pairs.

  \begin{block}{Problem: Parity of integer}
  \hspace{14pt}\underline{Input}: an integer

  \hspace{14pt}\underline{Output}: ``yes'' if the integer is odd. ``no'' otherwise.
  \end{block}

  \bigskip
  We can certainly write programs to solve this problem.

\end{frame}

\begin{frame}
  \frametitle{Limitations of computers: Decidability}

  \bigskip

  Other problems: PCP, halting problem

  \bigskip
  We will learn that these two problems are \alert{\textbf{undecidable}}.

  [i.e., whatever programs you write to solve these problems, they will
  be wrong on some input.]


  \bigskip
  \bigskip
  To be able to \alert{prove} these claims, we need precise
  \alert{mathematical models of computation}.

  \medskip
  In this class, we learn about \alert{automata}, \alert{grammar},
  and \alert{Turing machines}.

\end{frame}

\begin{frame}

  \frametitle{Complexity}

  Some problems are solvable but takes a long time, e.g. factoring.

  \bigskip
  We study the \alert{complexity} of these problems, i.e., the
  inherent hardness of the problems.
\end{frame}

\begin{frame}
  \frametitle{This course is useful!}

  \begin{itemize}
    \item Learn about limitations of computers
    \item Learn practical models: automata (calculators, doors,
    coke machines, microwaves, cruise control), grammars (compilers)
    \item Learn when to stop looking for better solutions to problems
    \item Learn \alert{abstract thinking} skills
  \end{itemize}
\end{frame}


\begin{frame}

In theory of computation, we are interested in \textbf{modelling}
computers, also known as \alert{machines}.

\bigskip
We first start with the simplest kind of machine,
\alert{finite automata}.

\bigskip
Their key feature is that they have a finite number of states. (So
they are also called ``finite state machines.'')
\end{frame}


\section[Finite Automata]{Finite Automata}

\subsection[example]{Example}
\begin{frame}
  \frametitle{Finite Automata}

  \bigskip  
  Finite automata models computers with very small memory.

  \medskip
  Examples: digital microwave, automatic doors, digital watches, fans,
  etc.
  \medskip

  \fbox{
    \small{
      \hspace{-20pt} \begin{tabular}{l|l}
    \begin{minipage}{1.5in}
  \[\UseTips
  \xymatrix {
    *++[o][F]{\mbox{\alert{closed}}} \ar@(ul,ur)^{\parbox{.8in}{\color{dgreen}{rear} \\ both
  \\ \color{dgreen}{neither} }} \ar@/^1pc/[r]^{\mbox{\color{dgreen}{front}}}
    & *++[o][F]{\mbox{\alert{open}}} \ar@/^2pc/[l]^{\mbox{\color{dgreen}{neither}}}
  \ar@(ul,ur)^{\parbox{.2in}{\color{dgreen}{front} \\ \color{dgreen}{rear} \\ \color{dgreen}{both} }}
  }
  \]
  \end{minipage} &
    \begin{minipage}{3in}
      \begin{tabular}{l|llll}
        \rowcolor{gray!40}
        \diagbox{\alert{state}}{\color{dgreen}{input}} & \color{dgreen}{neither} & \color{dgreen}{front} & \color{dgreen}{rear} & \color{dgreen}{both} \\
        \rowcolor{white}
        \gcc{\alert{closed}} & closed & open & closed & closed \\
        \gcc{\alert{open}} & closed & open & open & open 
      \end{tabular}
    \end{minipage}
  \end{tabular}}}
  
\end{frame}


\begin{frame}
  \frametitle{Another example of a finite automaton}

  \begin{block}{Problem: Parity of ones}

  \hspace{14pt}\underline{Input}: a string of zeros and ones

  \hspace{14pt}\underline{Output}: ``yes'' if there are odd number of ones. ``no'' otherwise.
  \end{block}

  \[\UseTips
  \xymatrix {
      *+\txt{\ } \ar[r] & *++[o][F]{q_1} \ar@(ul,ur)^{0}
  \ar@/^1pc/[r]^{1} & *++[o][F=]{q_2} \ar@(ul,ur)^{0}   \ar@/^1pc/[l]^{1}
  }
  \]
  
\end{frame}


\begin{frame}
  \frametitle{Another example of a finite automaton}

  \begin{tabular}{ll}
    \begin{minipage}{1.3in}
  Call this machine $M_1$ 
  \end{minipage}
    &
  \begin{minipage}{3in}
  \[\UseTips
  \xymatrix {
      *+\txt{\ } \ar[r] & *++[o][F]{q_1} \ar@(ul,ur)^{0}
  \ar@/^1pc/[r]^{1} & *++[o][F=]{q_2} \ar@(ul,ur)^{1}
  \ar@/^1pc/[r]^{0} & *++[o][F]{q_3} \ar@/^1pc/[l]^{0,1}
  }
  \]
  \end{minipage}
  \end{tabular}

  This is called a \alert{state diagram} of $M_1$.

  The \alert{states} of $M_1$ are $q_1, q_2, q_3$.
  
  The \alert{start state} is $q_1$.

  The \alert{accept state} is $q_2$.
  
  The \alert{transitions} are indicated by arrows going from
  state to state.

  The \alert{outputs} of $M_1$ are either \alert{accept} or \alert{reject}.

\end{frame}


\begin{frame}

  \[\UseTips
  \xymatrix {
      *+\txt{\ } \ar[r] & *++[o][F]{q_1} \ar@(ul,ur)^{0}
  \ar@/^1pc/[r]^{1} & *++[o][F=]{q_2} \ar@(ul,ur)^{1}
  \ar@/^1pc/[r]^{0} & *++[o][F]{q_3} \ar@/^1pc/[l]^{0,1}
  }
  \]

  Try using $M_1$ to process the strings 1101, 1, 01, 11, 0101010101,
  100, 0100, 0101000000, 0, 10, 101000.

  \medskip
  [Basically, it accepts anything ending with 1 and anything having even
  number of 0s following the last 1. It rejects other strings.]

\end{frame}


\subsection[definition]{Definition}

\begin{frame}
  \frametitle{Formal definition of a finite automaton}

  \begin{Definition}
    A \alert{finite
      automaton} is a 5-tuple $(Q,\Sigma, \delta, q_0, F)$ where
      \begin{enumerate}
	\item $Q$ is a finite set called the \alert{states},
	\item $\Sigma$ is a finite set called the \alert{alphabet},
	\item $\delta: Q\times \Sigma \rightarrow Q$ is the
	\alert{transition function},
	\item $q_0 \in Q$ is the \alert{start state},
	\item $F \subseteq Q$ is the \alert{set of accept states}.
      \end{enumerate}
  \end{Definition}

  \bigskip
  How would $M_1$ be formally described?

\end{frame}

\begin{frame}

  \[\UseTips
  \xymatrix {
      *+\txt{\ } \ar[r] & *++[o][F]{q_1} \ar@(ul,ur)^{0}
  \ar@/^1pc/[r]^{1} & *++[o][F=]{q_2} \ar@(ul,ur)^{1}
  \ar@/^1pc/[r]^{0} & *++[o][F]{q_3} \ar@/^1pc/[l]^{0,1}
  }
  \]

  The formal description of this machine is $(Q,\Sigma,\delta,q_1,F)$
  where
  \begin{enumerate}
    \item $Q = $ \only<2->{\alert{$\{q_1,q_2,q_3 \}$}}
    \item $\Sigma = $ \only<3->{\alert{$\{0,1\}$}}
    \item $\delta$ is this function:

      \only<4->{\alert{
	\begin{scriptsize}
	\begin{tabular}{l|ll} 
	  & 0 & 1 \\\hline
	  $q_1$ & $q_1$ & $q_2$ \\
	  $q_2$ & $q_3$ & $q_2$ \\
	  $q_3$ & $q_2$ & $q_2$
	\end{tabular}
	\end{scriptsize}}}
    \item The start state is \only<5->{\alert{$q_1$}}
    \item $F = $ \only<6->{\alert{$\{q_2\}$}}
  \end{enumerate}

\end{frame}

\begin{frame}
  \frametitle{What it means for a DFA to accept strings and to
  recognize a language}

  \begin{Definition}
    A DFA \alert{$M$ accepts a string} $w$ if, after executing on $w$
      as described, it ends up in an accept state.
  \end{Definition}

  \begin{Definition}
    A DFA \alert{$M$ recognizes a language} $L$ if
    \begin{enumerate}
    \item For every $w\in L$, $M$ accepts $w$
    \item For every $w\not\in L$, $M$ rejects $w$.
    \end{enumerate}
  \end{Definition}
\end{frame}


\begin{frame}
  \frametitle{More terminology}

  Let $M$ be a machine, and let $A$ be the set of all strings that a
  machine $M$ accepts. All these phrases mean the same thing.

  \begin{enumerate}
    \item The \alert{language} of machine $M = A$
    \item $M$ \alert{recognizes} $A$
    \item $L(M) = A$
    \item $M$ \alert{accepts} $A$
  \end{enumerate}

\end{frame}


\begin{frame}
  \frametitle{Observations}

  \begin{itemize}
    \item A machine may accept \underline{many} strings but can recognize only \underline{one}
      language.
      \bigskip

    \item A machine accepting \underline{no strings} is considered recognizing \underline{one
    language}, i.e., the empty language $\emptyset$.
      \bigskip

    \item If $L$ is a language, and $M$ is a DFA. It \alert{does
      not make sense} to say these things:
      \begin{itemize}
	\item the language $011$
	\item Run $L$ on input $011$.
	\item $L$ accepts $100$.
	\item The language accepted by $M$ is $011$.
      \end{itemize}
  \end{itemize}
\end{frame}


\note{
\begin{scriptsize}
  $A = \{ w | w \mbox{ contains at least one 1 and an even number of 0s
    following the last 1 }\}$

  Then, $L(M_1) = A$.

  $M_1$ recognizes $A$.
\end{scriptsize}}


\begin{frame}
  \frametitle{Formal definition of computation}
  
  We know how to \alert{compute} with a finite automaton, i.e., given a
  string $w$, we know how $M_1$ processes it.  Now we formalizes this
  process.

  \medskip
  Let $n\in \Z^+$, $M = (Q, \Sigma, \delta, q_0, F)$ be a finite
  automaton, and $w = w_1 w_2 \ldots w_n$ be a string over the
  alphabet $\Sigma$.

  \begin{Definition}
  \alert{$M$ accepts $w$} if a sequence of states $r_0, r_1, \ldots,
  r_n$ exists in $Q$ with the following three conditions:

  \begin{enumerate}
    \item[1.] [$M$ starts in the start state.] $r_0 = q_0$,
    \item[2.] [$M$ moves according to $\delta$.] $r_{i+1} = \delta(r_i, w_{i+1})$ for $i = 0,\ldots, n-1$,
    \item[3.] [$M$ ends up in an accept state.] $r_n \in F$.
  \end{enumerate}
  \end{Definition}

\end{frame}

\subsection[regular languages]{Regular languages}

\begin{frame}
  \frametitle{Regular languages}

  \begin{Definition}
    A language is called a \alert{regular language} if some finite
    automaton recognizes it.
  \end{Definition}
\end{frame}


\begin{frame}
  \frametitle{Designing finite automata}

  Given some description of the (regular) language, we are interested
  in constructing a finite automaton that recognizes it.
  
  \bigskip
  \underline{Examples}

  \begin{enumerate}
  \item The language of all strings with an odd number of
    1s.
  \item The language of all strings that contain the string 001 as a
  substring.
  \item $\{ w \; | \; w \mbox{ begins with a 1 and ends with a 0 } \}$
  \item $\{ w \; | \; w \mbox{ contains at least three 1s } \}$
  \item $\{ \varepsilon, 0 \}$
  \item $\emptyset$
  \item all strings except the empty string
  \end{enumerate}
  
\end{frame}

\begin{frame}
  \frametitle{More examples of regular languages}

  \begin{tabular}{ll}
  Alphabet $\Sigma$ &   $\{ 0 \}$\\
  Definition of $L_1$ &   $\{ 0, 00, 000, 0000, \ldots \}$ \\
  Sample strings in $L_1$ \ \ \  & $0, 00, 000, \ldots$ \\[1ex] \hline
  Alphabet $\Sigma$ &   $\{ 0 \}$\\
  Definition of $L_2$ &   $\{ 0^n$ for $n =1, 2, 3, \ldots \}$ \\
  Sample strings in $L_2$ & $0, 00, 000, \ldots$ \\[1ex] \hline
  Alphabet $\Sigma$ &   $\{ 0,1,2,3,4,5,6,7,8,9 \}$ \\
  Definition of $L_3$ &   $\{$ any finite string of letters that does not \\
  & start with the letter $0 \}$ \\
  Sample strings in $L_3$ & $1,2,3,4,5,6,7,8,9,10,11,12,\ldots$
  \end{tabular}
  
\end{frame}

\begin{frame}

  \begin{tabular}{lp{2.5in}}
  Alphabet $\Sigma$ &   $\{ 0,1,2,3,4,5,6,7,8,9 \}$ \\
  Definition of $L_4$ &   $\{$ any finite string of letters that, if it
  starts with the letter $0$, has no more letters after the first $\}$ \\
  Sample strings in $L_4$ & $0,1,2,3,4,5,6,7,8,9,10,11,12,\ldots$
  \\[1ex] \hline
  Alphabet $\Sigma$ &   $\{ a,b,c,d,e,\ldots,z \}$\\
  Definition of $L_5$ &   $\{$ all the words in the dictionary $\}$ \\
  Sample strings in $L_5$ \ \ \  & discussion, list, body, trick,
  $\ldots$ \\[1ex] \hline
  Alphabet $\Sigma$ &   $\{ a,b,c,d,e,\ldots,z, ., ! \}$\\
  Definition of $L_6$ &   $\{$ grammatical sequence of words in the dictionary $\}$ \\
  Sample strings in $L_6$ \ \ \  & The house is burning., Help me!, $\ldots$ 
  \end{tabular}
\end{frame}



\subsection[regular operations]{Regular operations}

\begin{frame}
  \frametitle{The regular operations}

  \begin{Definition}
    Let $A$ and $B$ be
      languages. We define the regular operations {union},
      {concatenation}, and {star} as follows:
      \begin{itemize}
	\item \alert{Union}: $A\cup B = \{ x \; | \; x \in A \mbox{ or } x \in B \}$
	\item \alert{Concatenation}: $A \circ B = \{ xy \; | \; x \in A \mbox{ and } y \in B \}$
	\item \alert{Star}: $A^* =\{ x_1 x_2\ldots x_k \; | \; k \geq 0 \mbox{ and each } x_i \in A \}$
      \end{itemize}
  \end{Definition}
  
  \begin{tabular}{lp{3in}}
    Union & take all the strings in both $A$ and $B$ and lump them together into one language. \\ \hline
    Concatenation & attach a string from $A$ in front of a string from $B$ in all possible ways to get strings in the new language. \\ \hline
    Star & attach any number of strings in $A$ together to get a string in the new language.
  \end{tabular}

\end{frame}

\begin{frame}
  \frametitle{Examples}
  Try regular operations on these languages:
  
  \begin{example}
    $L_1 = \{ \mbox{ alice, bob, carl } \} \;\;\;; \;\;\; L_2 = \{ 1,10 \} \;\;\;;\;\;\; L_3 = \{ \mbox{ ba, na } \}$
  \end{example}
  
  \medskip
  \begin{tabular}[t]{ll}
    \begin{minipage}{1in}
      \begin{itemize}
      \item $L_1 \cup L_1 = ?$ 
      \item $L_1 \cup L_2 = ?$ 
      \item $L_2 \cup L_1 = ?$ 
      \item $L_1 \circ L_2 = ?$ 
      \item $L_2 \circ L_1 = ?$ 
      \item $L_1 \circ L_1 = ?$ 
      \end{itemize}
    \end{minipage} &
    \begin{minipage}{3in}
      \begin{itemize}
      \item $L_1 \circ \{\varepsilon\} = ?$
      \item $L_1 \circ \emptyset = ?$
      \item $L_2^* = ?$ 
      \item $L_3^* = ?$ 
      \item Does $L_2^*$ contain $101?, \;10101?,\; 1001?$
      \item Does $L_3^*$ contain banana?
      \end{itemize}
    \end{minipage}
  \end{tabular}
\end{frame}

\subsection[closure]{Closure}

\begin{frame}
  \frametitle{Introducing closure}

  Consider the example languages we have seen:
  \begin{example}
    $L_1 = \{ \mbox{ alice, bob, carl } \} \;\;\;; \;\;\; L_2 = \{ 1,10 \} \;\;\;;\;\;\; L_3 = \{ \mbox{ ba, na } \}$
  \end{example}

  \begin{itemize}
    \item They are all regular languages. (Can you write DFAs for them?)
    \item Is $L_1 \circ L_2$ regular? What about $L_1 \cup L_2$? What about $L_3^*$?
  \end{itemize}
  \bigskip

  It turns out that, if you take any regular languages and apply a
  regular operation on them, you always get back a regular language.
\end{frame}

\begin{frame}
  \frametitle{Closure}

  \begin{Definition}
    A set is \alert{closed under an operation} $op$ if applying $op$
      to members of the set returns an object still in the set.
  \end{Definition}

  \bigskip
  
  \begin{Theorem}
    The class of regular languages is closed under the union,
      concatenation, and star operations.
  \end{Theorem}

  \bigskip
  If $A,B$ are regular languages, this means that $A\cup B$, $A \circ
  B$, and $A^*$ are all regular languages.
\end{frame}

\begin{frame}
  \frametitle{Proof?}

  \begin{Theorem}
    The class of regular languages is closed under union.
  \end{Theorem}  

  \bigskip
  How to prove this theorem?
\end{frame}

\section[Nondeterministic Finite Automata]{Nondeterministic Finite Automata}

\subsection[example]{Example}

\begin{frame}
  \begin{block}{}
    \[ L = \{ w \; \mid \; w \mbox{ has a 1 in the 2nd position from the
    right end } \} \]
  \end{block}

  We can write a DFA for this, but it is a bit of a pain. It is hard
  because we cannot rewind the input.

  \bigskip
  We can use nondeterminism to help us.

  \bigskip

  Suppose we can \alert{guess} where the 2nd position from the right
  end is. All we have to do once we make a guess is to \alert{verify}
  if our guess is correct.

\end{frame}

\begin{frame}
  \begin{block}{}
  \[ L = \{ w \; \mid \; w \mbox{ has a 1 in the 2nd position from the
    right end } \} \]
  \end{block}

  \[\UseTips
  \xymatrix {
      *+\txt{\ } \ar[r] & *++[o][F]{q_0} \ar@(ul,ur)^{0,1}
  \ar[r]^{1} & *++[o][F]{q_1} \ar[r]^{0,1} & *++[o][F=]{q_2}
  }
  \]  

  Try it on 01011: None of these accept, but it's our fault for making
  wrong guesses.

  \begin{itemize}
    \item $q_0 \stackrel{0}{\to} q_0 \stackrel{1}{\to} q_0 \stackrel{0}{\to} q_0 \stackrel{1}{\to} q_0 \stackrel{1}{\to} q_0$
    \item $q_0 \stackrel{0}{\to} q_0 \stackrel{1}{\to} q_1 \stackrel{0}{\to} q_2$
    \item $q_0 \stackrel{0}{\to} q_0 \stackrel{1}{\to} q_0 \stackrel{0}{\to} q_0 \stackrel{1}{\to} q_0 \stackrel{1}{\to} q_1$
  \end{itemize}

  \bigskip
  But there's a way to get the machine to accept!
 \[ q_0 \stackrel{0}{\to} q_0 \stackrel{1}{\to} q_0 \stackrel{0}{\to}
 q_0 \stackrel{1}{\to} q_1 \stackrel{1}{\to} q_2 \]

\end{frame}

\begin{frame}
  \begin{block}{}
  \[ L = \{ w \; \mid \; w \mbox{ has a 1 in the 2nd position from the
    right end } \} \]
  \end{block}

  \[\UseTips
  \xymatrix {
      *+\txt{\ } \ar[r] & *++[o][F]{q_0} \ar@(ul,ur)^{0,1}
  \ar[r]^{1} & *++[o][F]{q_1} \ar[r]^{0,1} & *++[o][F=]{q_2}
  }
  \]  

  \bigskip
  Try it on 1001: no way to get the machine to accept.

  e.g. $q_0 \stackrel{1}{\to} q_0 \stackrel{0}{\to} q_0
  \stackrel{0}{\to} q_0 \stackrel{1}{\to} q_1 $

  \bigskip
  Note that we can represent all the possible ways to process a string
  with the machine using a \alert{\textbf{tree}}.

  \bigskip
  The \alert{nodes} at each level in the tree can be thought of as
  your \alert{fingers} keeping track of all the possible states you
  can be at after having processed the input symbols upto that level.
\end{frame}


\subsection[definition]{Definition}

\begin{frame}

  \bigskip
  \begin{Definition}
    We say that \alert{an NFA $M$ accept an input string} $w$ if there
    is {\em some path\/} which can be followed on input $w$ and leads
    to an accept state.

    \medskip
    It \alert{rejects} $w$ if {\em all\/} paths that can be followed
	on input $w$ lead to rejection.
  \end{Definition}


  \bigskip
  \begin{Definition}
    We say that
      \alert{$M$ accepts a language $L$} if
      
      \begin{itemize}
	\item for every input $w\in L$ we have $M(w)$ accepts, and
      
	\item for every input $w\not\in L$, we have $M(w)$ rejects. 
      \end{itemize}

      We write, $L(M)$ for the language $L$ accepted by $M$.
      \end{Definition}

  
\end{frame}


\begin{frame}
  \frametitle{Another example}

  \begin{block}{}
  \[ L = \{  w \; \mid \; w \mbox{ contains 11 as a substring } \}\]
  \end{block}

  \[\UseTips
  \xymatrix {
      *+\txt{\ } \ar[r] & *++[o][F]{q_0} \ar@(ul,ur)^{0,1}
  \ar[r]^{1} & *++[o][F]{q_1} \ar[r]^{1} & *++[o][F=]{q_2}\ar@(ul,ur)^{0,1}
  }
  \]  

  \bigskip
  We can describe this formally also.
\end{frame}

\begin{frame}
  \frametitle{Formal definition of a nondeterministic finite automaton}

  \begin{Definition}
    A \alert{nondeterministic finite automaton} is a 5-tuple $(Q,\Sigma,
      \delta, q_0, F)$ where
      \begin{enumerate}
	\item $Q$ is a finite set called the \alert{states},
	\item $\Sigma$ is a finite set called the \alert{alphabet},
	\item $\delta: Q\times \Sigma_\varepsilon \rightarrow \mathcal{P}(Q)$ is the
	\alert{transition function},
	\item $q_0 \in Q$ is the \alert{start state},
	\item $F \subseteq Q$ is the \alert{set of accept states}.
      \end{enumerate}
      \end{Definition}

\end{frame}

\begin{frame}
  
  \[\UseTips
  \xymatrix {
      *+\txt{\ } \ar[r] & *++[o][F]{q_1} \ar@(ul,ur)^{0,1}
  \ar[r]^{0} & *++[o][F]{q_2} \ar[r]^{0,\varepsilon} & *++[o][F]{q_3}
  \ar[r]^{1} & *++[o][F=]{q_4} \ar@(ul,ur)^{0,1}
  }
  \]

  The formal description of this machine is $(Q,\Sigma,\delta,q_1,F)$
  where
  \begin{enumerate}
    \item $Q = $ \only<1->{\alert{$\{q_1,q_2,q_3,q_4 \}$}}
    \item $\Sigma = $\only<2->{\alert{$\{0,1\}$}}
    \item $\delta$ is this function:

      \only<3->{\alert{
	\begin{scriptsize}
	\begin{tabular}{l|lll} 
	  & 0 & 1 & $\varepsilon$ \\\hline
	  $q_1$ & $\{q_1,q_2\}$ & $\{q_1\}$ & $\emptyset$ \\
	  $q_2$ & $\{q_3\}$ & $\emptyset$ & $\{q_3\}$ \\
	  $q_3$ & $\emptyset$ & $\{q_4\}$ & $\emptyset$ \\
	  $q_4$ & $\{ q_4\}$ & $\{q_4\}$ & $\emptyset$ \\
	\end{tabular}
	\end{scriptsize}}}
    \item The start state is \only<4->{\alert{$q_1$}}
    \item $F = $ \only<5->{\alert{$\{q_4\}$}}
  \end{enumerate}
  
\end{frame}

\begin{frame}
  \frametitle{Formal definition of computation for NFA}
  
  Let $n\in \Z^+$, $M = (Q, \Sigma, \delta, q_0, F)$ be a finite
  automaton, and $w = w_1 w_2 \ldots w_n$ be a string over the
  alphabet $\Sigma$.

  \begin{Definition}
  \alert{$M$ accepts $w$} if a sequence of states $r_0, r_1, \ldots,
  r_n$ exists in $Q$ with the following three conditions:

  \begin{enumerate}
    \item[1.] [$M$ starts in the start state.] $r_0 = q_0$,
    \item[2.] [$r_{i+1}$ is one of the allowable next states from
    $r_i$ with input $w_{i+1}$.] \ \ \ \alert{$r_{i+1} \in \delta(r_i, w_{i+1})$}
    for $i = 0,\ldots, n-1$,
    \item[3.] [$M$ ends up in an accept state.] $r_n \in F$.
  \end{enumerate}
  \end{Definition}

\end{frame}


\subsection[NFA $\approx$ DFA]{Equivalence of NFAs and DFAs}

\begin{frame}
  \frametitle{Equivalence of NFAs and DFAs}

  \begin{Theorem}
    Every nondeterministic
      finite automaton has an equivalent deterministic finite automaton.
  \end{Theorem}

  \bigskip
  The proof of this theorem is called the \alert{subset construction}.

  Key observation: keep track of states with fingers on them.

  \begin{Corollary}
    A language is regular if
    and only if some nondeterministic finite automaton recognizes
    it.
  \end{Corollary}
\end{frame}

\subsection[closure]{Closure}

\begin{frame}
  \frametitle{Remember closure?}

  \begin{Theorem}
    The class of regular
    languages is closed under the union, concatenation, and star
    operations.
  \end{Theorem}


  \bigskip
  We can prove closure of regular languages under union,
  concatenation, and star very easily using \alert{NFAs}.
   
\end{frame}

\begin{frame}
  \frametitle{Closure of class of regular languages under union}

  \begin{Theorem}
    The class of regular languages is closed under the union
    operation.
  \end{Theorem}

  \begin{tiny}
  \begin{Proof}
    Let $L_1$ be a regular language recognized by an NFA $N_1 = (Q_1,\Sigma, \delta_1, q_1, F_1)$, and

    Let $L_2$ be a regular language recognized by an NFA $N_2 = (Q_2,\Sigma, \delta_2, q_2, F_2)$.

    Construct $N= (Q,\Sigma, \delta, q_0,F)$ for $L_1 \cup L_2$ as follows:
    \begin{enumerate}
      \item $Q = \{q_0\} \cup Q_1 \cup Q_2$ 
      \item $q_0$ is the start state of $N$.
      \item $F = F_1 \cup F_2$ 
      \item For any $q \in Q$ and $a \in \Sigma\cup \{\varepsilon\}$,
        \[ \delta(q,a) = 
        \begin{cases} 
          \delta_1(q,a) & \text{ if } q \in Q_1 \\
          \delta_2(q,a) & \text{ if } q \in Q_2 \\
          \{ q_1, q_2 \} & \text{ if } q =q_0 \text{ and } a = \varepsilon \\
          \emptyset & \text{ if } q =q_0 \text{ and } a \neq \varepsilon 
        \end{cases} 
        \]
    \end{enumerate}

  \end{Proof}
  \end{tiny}%
\end{frame}

\end{document}
